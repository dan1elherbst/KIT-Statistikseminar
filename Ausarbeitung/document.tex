\documentclass[]{article}

\usepackage[margin=1.5in]{geometry}
\usepackage[utf8]{inputenc} % this is needed for umlauts
\usepackage[ngerman]{babel} % this is needed for umlauts
\usepackage[T1]{fontenc}    % this is needed for correct output of umlauts in pdf
\usepackage{amsmath}
\usepackage{amssymb}
\usepackage{bm}
\usepackage{amsthm}
\usepackage{hyperref}
\usepackage{mdframed}
\usepackage{xcolor}

\newcommand{\Pb}{\mathbb{P}}
\newcommand{\E}{\mathbb{E}}
\newcommand{\R}{\mathbb{R}}
\newcommand{\N}{\mathbb{N}}
\newcommand{\X}{\mathbf{X}}
\newcommand{\Y}{\mathbf{Y}}
\newcommand{\T}{\mathbf{\Theta}}
\newcommand{\muu}{\bm{\mu}}
\newcommand{\Ssigma}{\mathbf{\Sigma}}
\newcommand{\vv}{\mathbf{v}}
\newcommand{\uu}{\mathbf{u}}
\newcommand{\C}{\mathbf{C}}
\newcommand{\rk}{\mathrm{rk}}
\newcommand{\A}{\mathbf{A}}
\newcommand{\B}{\mathbf{B}}
\newcommand{\Ggamma}{\mathbf{\Gamma}}
\newcommand{\tr}{\mathrm{tr}}

\newmdtheoremenv[backgroundcolor=black!5, innertopmargin=-2pt]{definition}{Definition}[section]
\newmdtheoremenv[backgroundcolor=black!5, innertopmargin=-2pt]{theorem}[definition]{Satz}
\newmdtheoremenv[backgroundcolor=black!5, innertopmargin=-2pt]{lemma}[definition]{Lemma}
\newtheorem*{remark}{Bemerkung}
\DeclareMathOperator*{\argmin}{argmin}

%opening
\title{Reduced-Rank-Regression}
\author{Daniel Herbst}
\date{10. Mai 2021}

\begin{document}

\maketitle

\begin{abstract}
Nachdem in der vorigen Präsentation das multivariate Regressionsmodell mit festen Inputvariablen vorgestellt wurde und 
bereits Modelle betrachtet wurden, bei denen die Regressionskoeffizienten gewisse (lineare) Nebenbedingungen
erfüllen, wollen wir nun eine ähnliche Situation betrachten, bei der wir die Inputvariablen allerdings als
zufällig annehmen und den Rang der Regressionskoeffizientenmatrix einschränken. Hierbei handelt es sich dann um das
Reduced-Rank-Regressionsmodell, das unter anderem auch einige Techniken der multivariaten Statistik, etwa zur Dimensionsreduktion, 
verallgemeinert. Im Wesentlichen folgt der Vortrag 
\cite[Kapitel 6.3]{Iz08}.
\end{abstract}

\section{Einleitung}
\label{Einleitung}
Unsere Ausgangssituation ist die Folgende:
$$\X := (X_1, \dots, X_r)^\top \quad \text{und} \quad \Y := (Y_1, \dots, Y_s)^\top \text{,}$$
seien Zufallsvektoren mit gemeinsamer Verteilung $\Pb^{(\X, \Y)}$, wobei $r, s \in \N$ mit $s \leq r$,
und ferner definieren wir folgende Schreibweisen für die entsprechenden Erwartungwerte und Kovarianzmatrizen:
$$ \muu_\X := \E\X, \quad \muu_\Y := \E\Y \quad \text{und} \quad \begin{pmatrix}
	\Ssigma_{\X\X} & \Ssigma_{\X\Y} \\
	\Ssigma_{\Y\X} & \Ssigma_{\Y\Y}
\end{pmatrix} := \Sigma \biggl(\begin{pmatrix}
\X \\
\Y
\end{pmatrix}\biggr).$$
Abgesehen von $\Pb^\X \ll \lambda^r$ und $\Pb^\Y \ll \lambda^s$, d.h. der Stetigkeit der Zufallsvektoren $\X$ und $\Y$, möchten wir zunächst keine
weiteren Annahmen über die Verteilungen von $\X$ und $\Y$ treffen.

\section{Klassisches multivariates Regressionsmodell mit zufälliger Inputvariable}

Das klassische multivariate Regressionsmodell mit zufälliger Inputvariable ist von folgender Form: Für $\X$ und $\Y$ gelte
\[ \Y = \muu + \T \X + \mathcal{E} \text{,} \label{eq:2.1} \tag{2.1}\]
wobei $\muu \in \R^s$ und $\T \in \R^{s \times r}$ unbekannte Parameter seien sowie $\mathcal{E}$ ein (nicht beobachtbarer) $s$-dimensionaler zufälliger Fehler ist mit Erwartungswert $\E \mathcal{E} = 0$ und Kovarianzmatrix $\Ssigma_{\mathcal{E} \mathcal{E}}$. Zudem werden $\X$ und
$\mathcal{E}$ als unabhängig vorausgesetzt.
\\

Ausgehend von der Situation aus der Einleitung wollen wir nun optimale $\muu$ und $\T$ finden in dem Sinne, dass
$$W(\muu, \T) := \E[(\Y - \muu - \T \X)(\Y - \muu - \T \X)^\top] \in \R^{s \times s}$$ in der Spektralnorm
$$ \| \mathbf{A} \|_2 := \max_{\| \vv \|_2 = 1} \| \mathbf{A} \vv \|_2 \text{,} \quad \mathbf{A} \in \R^{s \times s} $$ 
minimiert wird.

Für symmetrische, positiv semidefinite Matrizen $\mathbf{A}$ lässt sich (etwa mit dem Spektralsatz) zeigen, dass
$$ \| \mathbf{A} \|_2 = \max_{\| \vv \|_2 = 1} \vv^\top \mathbf{A} \vv = \lambda_{\max} (\mathbf{A}) \text{,} $$
wobei $\lambda_{\max} (\mathbf{A})$ der größte Eigenvektor von $\mathbf{A}$ sei. Da $W(\muu, \T)$ für alle möglichen $\muu$, $\T$ symmetrisch
positiv semidefinit ist, werden wir diese Darstellung der Spektralnorm auch zur Minimierung von $\| W(\muu, \T) \|_2$  benutzen.

\begin{theorem}
	\label{thm:mr}
	Seien $\X$ und $\Y$ wie in der \nameref{Einleitung} und zudem $\Ssigma_{\X\X}$ nichtsingulär, so ist
	$$ \argmin_{(\muu, \T)} \| W(\muu, \T) \|_2 = (\muu_\Y - \T \muu_\X, \Ssigma_{\Y\X} \Ssigma_{\X\X}^{-1}) =: (\muu_{\min}, \T_{\min}) $$
	mit 
	$$W(\muu_{\min}, \T_{\min}) = \Ssigma_{\Y\Y} - \Ssigma_{\Y\X} \Ssigma_{\X\X}^{-1} \Ssigma_{\X\Y} \text{.}$$
	Ferner gilt für $\mathcal{E} := \Y - \muu_{\min} - \T_{\min} \X$, dass 
	$$\E \mathcal{E} = 0 \quad \text{und} \quad \E[\mathcal{E}(\X - \muu_\X)^\top] = 0 \text{,}$$
	d.h. $\X$ und $\mathcal{E}$ sind unkorreliert.
\end{theorem} 

\begin{proof}
	Mit $\X_c := \X - \muu_\X$, $\Y_c := \Y - \muu_\Y$ gilt für alle $\T \in \R^{s \times r}$, $\muu \in \R^s$:
	\begin{align*}
		W(\muu, \T) ={}& \E[(\Y - \muu - \T \X)(\Y - \muu - \T \X)^\top] \\
		={}& \E[(\Y_c - \T \X_c + (\muu_\Y - \T \muu_\X - \muu))(\Y_c - \T \X_c + (\muu_\Y - \T \muu_\X - \muu))^\top] \\
		={}& \E[\Y_c \Y_c^\top - \Y_c \X_c^\top \T^\top + \Y_c (\muu_\Y - \T \muu_\X - \muu)^\top \\
		& \quad - \T \X_c \Y_c^\top + \T \X_c \X_c^\top \T^\top - \T \X_c (\muu_\Y - \T \muu_\X - \muu)^\top \\
		& \quad + (\muu_\Y - \T \muu_\X - \muu) \Y_c^\top - (\muu_\Y - \T \muu_\X - \muu) \X_c^\top \T^\top \\
		& \quad + (\muu_\Y - \T \muu_\X - \muu)(\muu_\Y - \T \muu_\X - \muu)^\top] \\
		={}& \Ssigma_{\Y\Y} - \Ssigma_{\Y\X} \T^\top - \T \Ssigma_{\X\Y} + \T \Ssigma_{\X\X} \T^\top \\
		& \quad + (\muu_\Y - \T \muu_\X - \muu)(\muu_\Y - \T \muu_\X - \muu)^\top \\
		={}& \Ssigma_{\Y\Y} - \Ssigma_{\Y\X} \Ssigma_{\X\X}^{-1} \Ssigma_{\X\Y} \\
		& \quad + (\Ssigma_{\Y\X} \Ssigma_{\X\X}^{-1/2} - \T \Ssigma_{\X\X}^{1/2})(\Ssigma_{\Y\X} \Ssigma_{\X\X}^{-1/2} - \T \Ssigma_{\X\X}^{1/2})^\top \\
		& \quad + (\muu_\Y - \T \muu_\X - \muu)(\muu_\Y - \T \muu_\X - \muu)^\top \text{,}
	\end{align*}
	und mit $$A(\T) := \Ssigma_{\Y\X} \Ssigma_{\X\X}^{-1/2} - \T \Ssigma_{\X\X}^{1/2} \quad \text{,} \quad B(\muu, \T) := \muu_\Y - \T \muu_\X - \muu$$ gilt für $\vv \in \R^s$, $\| \vv \|_2 = 1$:
	\begin{align*}
		\vv^\top W(\muu, \T) \vv ={}& \vv^\top (\Ssigma_{\Y\Y} - \Ssigma_{\Y\X} \Ssigma_{\X\X}^{-1} \Ssigma_{\X\Y}) \vv
		+ \vv^\top A(\T)^\top A(\T) \vv + \vv^\top B(\muu, \T)^\top B(\muu, \T) \vv \\
		={}& \vv^\top (\Ssigma_{\Y\Y} - \Ssigma_{\Y\X} \Ssigma_{\X\X}^{-1} \Ssigma_{\X\Y}) \vv
		+ \| A(\T) \vv \|_2 + \| B(\muu, \T) \vv \|_2 \\
		\geq{}& \vv^\top (\Ssigma_{\Y\Y} - \Ssigma_{\Y\X} \Ssigma_{\X\X}^{-1} \Ssigma_{\X\Y}) \vv
	\end{align*}
	und daher ist
	$$ \| W(\muu, \T) \|_2 \geq \| \Ssigma_{\Y\Y} - \Ssigma_{\Y\X} \Ssigma_{\X\X}^{-1} \Ssigma_{\X\Y} \|_2 \text{,}$$
	mit Gleichheit falls $A(\T) = B(\muu, \T) = 0$, was der Fall ist, wenn
	$$ \muu = \muu_\Y - \T \muu_\X =: \muu_{\min} \quad \text{und} \quad \T = \Ssigma_{\Y\X} \Ssigma_{\X\X}^{-1} =: \T_{\min} \text{.}$$
	Man erhält nun
	\begin{align*}
		\mathcal{E} ={}& \Y - \muu_{\min} - \T_{\min} \X \\
		={}& \Y - (\muu_\Y - (\Ssigma_{\Y\X} \Ssigma_{\X\X}^{-1}) \muu_\X) - (\Ssigma_{\Y\X} \Ssigma_{\X\X}^{-1}) \X \\
		={}& \Y_c - \Ssigma_{\Y\X} \Ssigma_{\X\X}^{-1} \X_c
	\end{align*}
	und es gilt offensichtlich $\E \mathcal{E} = 0$ sowie
	$$\E[\mathcal{E} \X_c^\top] = \E[(\Y_c - \Ssigma_{\Y\X} \Ssigma_{\X\X}^{-1} \X_c) \X_c^\top] = \Ssigma_{\Y\X} - \Ssigma_{\Y\X}\Ssigma_{\X\X}^{-1} \Ssigma_{\X\X} = 0 \text{,}$$
	d.h. $\X$ und $\mathcal{E}$ sind unkorreliert.
\end{proof}

\section{Reduced-Rank-Regressionsmodell}

Wir betrachten nun das folgende Regressionsmodell
\[\Y = \muu + \C \X + \mathcal{E} \text{,} \label{eq:3.1} \tag{3.1}\]
bei dem $\muu \in \R^s$ und $\C \in \R^{s \times r}$ unbekannte Parameter sind sowie $\mathcal{E}$ ein nicht beobachtbarer, zufälliger Fehler mit $\E \mathcal{E} = 0$ und Kovarianzmatrix $\Ssigma_{\mathcal{E} \mathcal{E}}$, wobei $\X$ und $\mathcal{E}$ unabhängig seien. 

Im Gegensatz zu \eqref{eq:2.1}, wo wir über $\T$ keine weiteren Annahmen treffen wollten, soll es nun zusätzlich ein gewisses $t$ geben mit
$$ \rk \, \C \leq t \leq s \text{.}$$
In diesem Fall gibt es Matrizen $\A \in \R^{s \times t}$ und $\B \in \R^{t \times r}$ mit $\C = \A \B$, d.h. das Modell \eqref{eq:3.1}
lässt sich nun schreiben als
\[\Y = \muu + \A \B \X + \mathcal{E} \text{,} \label{eq:3.2} \tag{3.2}\]
wobei $\muu$, $\A$ und $\B$ nun die unbekannten Parameter sind.

Auch hier möchten wir für $\X$ und $\Y$ wie in Satz~\ref{thm:mr} $\muu \in \R^s$, $\A \in \R^{s \times t}$ und $\B \in \R^{t \times r}$ finden, die $\Y - \muu - \A \B \X$ auf eine gewisse Art und Weise minimieren, gehen hier aber etwas anders vor:
Für eine symmetrische, positiv definite Gewichtsmatrix $\Ggamma \in \R^{s \times s}$ wollen wir nun 
$$ W_{t, \Ggamma}(\muu, \A, \B) := \E[(\Y - \muu - \A \B \X)^{\top} \Ggamma (\Y - \muu - \A \B \X)]$$
über $\muu$, $\A$ und $\B$ minimieren.

\begin{theorem}
	\label{thm:rrr}
	Seien $\X$ und $\Y$ wie in der \nameref{Einleitung}, $\Ssigma_{\X\X}$ nichtsingulär und $t \leq \rk \, \Ssigma_{\Y\X} \leq s$. Dann ist
	$$\argmin_{(\muu, \A, \B)} W_{t, \Ggamma}(\muu, \A, \B) = (\muu^{(t)}_{\min}, \A^{(t)}_{\min}, \B^{(t)}_{\min})$$ mit
	\begin{alignat*}{3}
		&\A^{(t)}_{\min}   &&:= \Ggamma^{-1/2} \mathbf{U}_t                                       &&\quad \in \R^{s \times t} \\
		&\B^{(t)}_{\min}   &&:= \mathbf{U}_t^\top \Ggamma^{1/2} \Ssigma_{\Y\X}\Ssigma_{\X\X}^{-1} &&\quad \in \R^{t \times r} \\
		&\muu^{(t)}_{\min} &&:= \muu_\Y - \A^{(t)}_{\min} \B^{(t)}_{\min} \muu_\X                 &&\quad \in \R^s \text{,}
	\end{alignat*}
	wobei $\mathbf{U}_t := (\uu_1,\dots, \uu_t) \in \R^{s \times t}$ und $\uu_1,\dots, \uu_t$ orthonormale Eigenvektoren zu den Eigenwerten $\lambda_1 \geq \dots \geq \lambda_t > 0$ von $\Ggamma^{1/2} \Ssigma_{\Y\X} \Ssigma_{\X\X}^{-1} \Ssigma_{\X\Y} \Ggamma^{1/2}$ seien.
	Ferner gilt
	$$W_{t, \Ggamma}(\muu^{(t)}_{\min}, \A^{(t)}_{\min}, \B^{(t)}_{\min}) = \tr(\Ssigma_{\Y\Y}) \tr(\Ggamma) - \sum_{i=1}^{t} \lambda_i \text{.}$$
\end{theorem}

Für den Beweis benötigen wir zunächst einen Hilfssatz aus der Linearen Algebra:

\begin{lemma}[Satz von Eckart-Young]
	\label{thm:eckart-young}
	Seien $\A, \B \in \R^{m \times n}$ und $b := \rk \, \B \leq \rk \, \A =: r$, und sei
	$\lambda_j(\C)$ für eine reelle symmetrischen Matrix $\C$ deren $j$-größter Eigenwert.
	Dann gilt:
	$$\lambda_j((\A - \B)(\A - \B)^\top) \geq \lambda_{j+b}(\A \A^\top)$$
	mit Gleichheit für
	$$\A_b := \sum_{i=1}^{b} \lambda_i^{1/2} \mathbf{u}_i \mathbf{v}_i^\top \text{,}$$
	wobei $\lambda_i := \lambda_i(\A \A^\top)$ und $\mathbf{u}_i$ bzw. $\mathbf{v}_i$ jeweils orthonormale Eigenvektoren von $\A \A^\top$ bzw. 
	$\A^\top \A$ zu $\lambda_i$ seien.
\end{lemma}

\begin{proof}
	Der Beweis nutzt die Singulärwertzerlegung und folgt etwa direkt aus \cite[Satz 4.6]{BZ21}.
\end{proof}

\begin{remark}
	Insbesondere zeigt Lemma~\ref{thm:eckart-young}, dass das dort definierte $\A_b$ den Abstand von $\A$ unter allen Rang-$b$-Matrizen in $\R^{m \times n}$ in der Spektralnorm minimiert.
\end{remark}

\begin{proof}[Beweis von Satz~\ref{thm:rrr}]
	Seien $\muu \in \R^s$, $\A \in \R^{s \times t}$, $\B \in \R^{t \times r}$ und setze $\X_c := \X - \muu_\X$, $\Y_c = \Y - \muu_\Y$, $\C := \A \B$. 
	Dann gilt:
	\begin{align*}
		W_{t, \Ggamma}(\muu, \C) ={}& \E[(\Y - \muu - \C \X)^\top \Ggamma (\Y - \muu - \C \X)] \\
		={}& \E[(\Y_c - \C \X_c + (\muu_\Y - \C \muu_\X - \muu))^\top \Ggamma (\Y_c - \C \X_c + (\muu_\Y - \C \muu_\X - \muu))] \\
		={}& \E[(\Y_c - \C \X_c)^\top \Ggamma (\Y_c - \C \X_c)] \\
		   & \quad + \underbrace{(\muu_\Y - \C \muu_\X - \muu)^\top \Ggamma (\muu_\Y - \C \muu_\X - \muu)}_{\geq 0 \quad \text{(da } \Ggamma \text{ symmetrisch positiv definit)}} \label{eq:3.3} \tag{3.3}
	\end{align*} 
	Unser Ziel ist es nun, \eqref{eq:3.3} zu minimieren, indem wir zunächst den ersten Term nach $\C$ minimieren. Für das so festgelegte $\C_{\min}$ lässt sich $\muu_{\min} := \muu_\Y - \C_{\min} \muu_\X$ aber immer noch so wählen, dass der zweite Term zu $0$ wird, was impliziert, dass die so bestimmte Kombination $(\muu_{\min}, \C_{\min})$ tatsächlich $W_{t, \Ggamma}(\muu, \C)$ minimiert. \\

	Mit $\Ssigma_{\X\X}^\ast := \Ssigma_{\X\X}$, $\Ssigma_{\Y\Y}^\ast := \Ggamma^{1/2} \Ssigma_{\Y\Y} \Ggamma^{1/2}$, $\Ssigma_{\Y\X}^\ast := \Ggamma^{1/2} \Ssigma_{\Y\X}$, $\Ssigma_{\X\Y}^\ast := \Ssigma_{\X\Y} \Ggamma^{1/2}$ und $\C^\ast := \Ggamma^{1/2} \C$ ist
	\begin{align*}
		\E[(\Y_c - \C \X_c)^\top \Ggamma (\Y_c - \C \X_c)] ={}& \E[\Y_c^\top \Ggamma \Y_c - \X_c^\top \C^\top \Ggamma \Y_c - \Y_c^\top \Ggamma \C
		\X_c + \X_c^\top \C^\top \Ggamma \C \X_c] \\
		={}& \tr(\Ssigma_{\Y\Y}^\ast - \C^\ast \Ssigma_{\X\Y}^\ast - \Ssigma_{\Y\X}^\ast \C^{\ast
			\top} + \C^\ast \Ssigma_{\X\X}^\ast \C^{\ast \top}) \\
		={}& \tr(\Ssigma_{\Y\Y}^\ast - \Ssigma_{\Y\X}^\ast \Ssigma_{\X\X}^{\ast -1}
		\Ssigma_{\X\Y}^\ast) \\
		&+ \tr( \; (\C^\ast \Ssigma_{\X\X}^{\ast 1/2} - \Ssigma_{\Y\X}^\ast \Ssigma_{\X\X}^{\ast -1/2}) \\
		& \qquad \cdot (\C^\ast \Ssigma_{\X\X}^{\ast 1/2} - \Ssigma_{\Y\X}^\ast \Ssigma_{\X\X}^{\ast -1/2})^\top \; ) \text{.} \label{eq:3.4} \tag{3.4}                                                   
	\end{align*}
    Sei nun 
    $$k := \rk \, \Ssigma_{\Y\X} = \rk \, \Ssigma_{\Y\X}^\ast \Ssigma_{\X\X}^{\ast -1/2} \geq t$$ 
    und seien $\uu_i$ orthonormale Eigenvektoren zum jeweils $i$-größten Eigenwert $\lambda_i$ von
	$$(\Ssigma_{\Y\X}^\ast \Ssigma_{\X\X}^{\ast -1/2})(\Ssigma_{\Y\X}^\ast \Ssigma_{\X\X}^{\ast -1/2})^\top = \Ssigma_{\Y\X}^\ast \Ssigma_{\X\X}^{\ast -1} \Ssigma_{\X\Y}^\ast = \Ggamma^{1/2} \Ssigma_{\Y\X} \Ssigma_{\X\X}^{-1} \Ssigma_{\X\Y} \Ggamma^{1/2}$$
	für $1 \leq i \leq k$ und 
	$$\vv_i := \lambda_i^{-1/2} \Ssigma_{\X\X}^{\ast -1/2} \Ssigma_{\X\Y}^\ast \uu_i = \lambda_i^{-1/2} \Ssigma_{\X\X}^{-1/2} \Ssigma_{\X\Y} \Ggamma^{1/2} \uu_i$$
	zugehörige orthonormale Eigenvektoren von $(\Ssigma_{\Y\X}^\ast \Ssigma_{\X\X}^{\ast -1/2})^\top(\Ssigma_{\Y\X}^\ast \Ssigma_{\X\X}^{\ast -1/2})$, so lässt sich der Satz von Eckart-Young (Lemma~\ref{thm:eckart-young}) wie folgt anwenden: Der zweite Term wird minimal für 
	$$\C^\ast \Ssigma_{\X\X}^{\ast 1/2} = \Ggamma^{1/2} \C \Ssigma_{\X\X}^{1/2} = \sum_{i=1}^{t} \lambda_i^{1/2} \uu_i \vv_i^\top$$
	(denn die Spur ist als Summe der Eigenwerte monoton steigend in jedem Eigenwert), also folgt, dass
	\begin{align*}
		\C^{(t)}_{\min} :={}& \Ggamma^{-1/2} \biggl( \sum_{i=1}^{t} \lambda_i^{1/2} \uu_i \vv_i^\top \biggr) \Ssigma_{\X\X}^{-1/2} \\
		                 ={}& \Ggamma^{-1/2} \biggl( \sum_{i=1}^{t} \uu_i \uu_i^\top \biggr) \Ggamma^{1/2} \Ssigma_{\Y\X} \Ssigma_{\X\X}^{-1}
	\end{align*}
	den ersten Summand von \eqref{eq:3.3} über alle Matrizen $\C \in \R^{s \times r}$ von Rang $\leq t$ minimiert.
	Daher wird $W_{t, \Ggamma}(\muu, \A, \B)$ nun minimal für
	\begin{alignat*}{3}
		&\A^{(t)}_{\min}   &&= \Ggamma^{-1/2} \mathbf{U}_t                                       &&\quad \in \R^{s \times t} \\
		&\B^{(t)}_{\min}   &&= \mathbf{U}_t^\top \Ggamma^{1/2} \Ssigma_{\Y\X}\Ssigma_{\X\X}^{-1} &&\quad \in \R^{t \times r} \\
		&\muu^{(t)}_{\min} &&= \muu_\Y - \A^{(t)}_{\min} \B^{(t)}_{\min} \muu_\X                 &&\quad \in \R^s \text{,}
	\end{alignat*}
	wobei $\mathbf{U}_t := (\uu_1,\dots, \uu_t) \in \R^{s \times t}$.
	Mit \eqref{eq:3.3}, \eqref{eq:3.4} ist das dadurch angenommene Minimum dann
	\begin{align*}
		W_{t, \Ggamma}(\muu^{(t)}_{\min}, \A^{(t)}_{\min}, \B^{(t)}_{\min}) ={}& \tr( \Ggamma^{1/2} \Ssigma_{\Y\Y} \Ggamma^{1/2} - \Ggamma^{1/2} \Ssigma_{\Y\X} \Ssigma_{\X\X}^{-1} \Ssigma_{\X\Y} \Ggamma^{1/2}) \\
																			   &+ \tr( (\sum_{i=t+1}^{k} \lambda_i^{1/2} \uu_i \vv_i^\top) (\sum_{i=t+1}^{k} \lambda_i^{1/2} \uu_i \vv_i^\top)^\top \; ) \\
																			={}& \tr(\Ssigma_{\Y\Y}) \tr(\Ggamma) - \tr(\Ggamma^{1/2} \Ssigma_{\Y\X} \Ssigma_{\X\X}^{-1} \Ssigma_{\X\Y} \Ggamma^{1/2}) + \sum_{i=t+1}^{k} \lambda_i \\
																			={}& \tr(\Ssigma_{\Y\Y}) \tr(\Ggamma) - \sum_{i=1}^{t} \lambda_i	\text{.} \qedhere										
	\end{align*}
\end{proof}

\newpage

\begin{thebibliography}{9}
	\bibitem[Iz08]{Iz08} 
	Izenman, Alan Julian.
	\textit{Modern Multivariate Statistical Techniques: Regression, Classification, and Manifold Learning}. 
	Springer Texts in Statistics, Springer-Verlag New York, 2008.
	\bibitem[BZ21]{BZ21}
	Bandeira, Afonso S. und Zhivotovskiy, Nikita.
	\textit{Lecture Notes for Mathematics of Machine Learning}. 2021.
	\url{https://metaphor.ethz.ch/x/2021/fs/401-2684-00L/sc/Math_of_ML_Lecture_Notes.pdf}
\end{thebibliography}



\end{document}
